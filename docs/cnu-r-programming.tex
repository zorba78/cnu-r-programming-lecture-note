% Options for packages loaded elsewhere
\PassOptionsToPackage{unicode}{hyperref}
\PassOptionsToPackage{hyphens}{url}
\PassOptionsToPackage{dvipsnames,svgnames*,x11names*}{xcolor}
%
\documentclass[
  11pt,
]{krantz}
\usepackage{lmodern}
\usepackage{amssymb,amsmath}
\usepackage{ifxetex,ifluatex}
\ifnum 0\ifxetex 1\fi\ifluatex 1\fi=0 % if pdftex
  \usepackage[T1]{fontenc}
  \usepackage[utf8]{inputenc}
  \usepackage{textcomp} % provide euro and other symbols
\else % if luatex or xetex
  \usepackage{unicode-math}
  \defaultfontfeatures{Scale=MatchLowercase}
  \defaultfontfeatures[\rmfamily]{Ligatures=TeX,Scale=1}
\fi
% Use upquote if available, for straight quotes in verbatim environments
\IfFileExists{upquote.sty}{\usepackage{upquote}}{}
\IfFileExists{microtype.sty}{% use microtype if available
  \usepackage[]{microtype}
  \UseMicrotypeSet[protrusion]{basicmath} % disable protrusion for tt fonts
}{}
\makeatletter
\@ifundefined{KOMAClassName}{% if non-KOMA class
  \IfFileExists{parskip.sty}{%
    \usepackage{parskip}
  }{% else
    \setlength{\parindent}{0pt}
    \setlength{\parskip}{6pt plus 2pt minus 1pt}}
}{% if KOMA class
  \KOMAoptions{parskip=half}}
\makeatother
\usepackage{xcolor}
\IfFileExists{xurl.sty}{\usepackage{xurl}}{} % add URL line breaks if available
\IfFileExists{bookmark.sty}{\usepackage{bookmark}}{\usepackage{hyperref}}
\hypersetup{
  pdftitle={통계 프로그래밍 언어},
  pdfauthor={한국한의학연구원, 구본초},
  colorlinks=true,
  linkcolor=Maroon,
  filecolor=Maroon,
  citecolor=Blue,
  urlcolor=Blue,
  pdfcreator={LaTeX via pandoc}}
\urlstyle{same} % disable monospaced font for URLs
\usepackage{longtable,booktabs}
% Correct order of tables after \paragraph or \subparagraph
\usepackage{etoolbox}
\makeatletter
\patchcmd\longtable{\par}{\if@noskipsec\mbox{}\fi\par}{}{}
\makeatother
% Allow footnotes in longtable head/foot
\IfFileExists{footnotehyper.sty}{\usepackage{footnotehyper}}{\usepackage{footnote}}
\makesavenoteenv{longtable}
\usepackage{graphicx,grffile}
\makeatletter
\def\maxwidth{\ifdim\Gin@nat@width>\linewidth\linewidth\else\Gin@nat@width\fi}
\def\maxheight{\ifdim\Gin@nat@height>\textheight\textheight\else\Gin@nat@height\fi}
\makeatother
% Scale images if necessary, so that they will not overflow the page
% margins by default, and it is still possible to overwrite the defaults
% using explicit options in \includegraphics[width, height, ...]{}
\setkeys{Gin}{width=\maxwidth,height=\maxheight,keepaspectratio}
% Set default figure placement to htbp
\makeatletter
\def\fps@figure{htbp}
\makeatother
\setlength{\emergencystretch}{3em} % prevent overfull lines
\providecommand{\tightlist}{%
  \setlength{\itemsep}{0pt}\setlength{\parskip}{0pt}}
\setcounter{secnumdepth}{5}
\usepackage{kotex}
\usepackage{booktabs}
\usepackage{placeins}
\usepackage{enumerate}
\usepackage{amssymb}
\usepackage{amsmath}
\usepackage{mathtools}
\usepackage{float}
% \usepackage{setspace} \doublespacing
\usepackage{relsize}
\usepackage{bigints}
\usepackage{bm}
\usepackage{amsmath}
% \usepackage{titlesec}
\usepackage{lipsum}
\usepackage{longtable}
 \usepackage[font=small,labelfont=bf]{caption}
\usepackage{dcolumn}
\usepackage{array}
\usepackage{gensymb}
\usepackage{makecell}
\usepackage{multirow}
\usepackage{natbib}
\usepackage{rotating}

\renewcommand\theadalign{cb}
\renewcommand\theadfont{\bfseries}
\renewcommand\theadgape{\Gape[4pt]}
\renewcommand\cellgape{\Gape[4pt]}
\DeclareMathAlphabet{\mathpzc}{OT1}{pzc}{m}{it}

\renewcommand\theadalign{cb}
\renewcommand\theadfont{\bfseries}
\renewcommand\theadgape{\Gape[4pt]}
\renewcommand\cellgape{\Gape[4pt]}

\newcolumntype{L}[1]{>{\raggedright\let\newline\\
\arraybackslash\hspace{0pt}}m{#1}}
\newcolumntype{C}[1]{>{\centering\let\newline\\
\arraybackslash\hspace{0pt}}m{#1}}
\newcolumntype{R}[1]{>{\raggedleft\let\newline\\
\arraybackslash\hspace{0pt}}m{#1}}
\newcolumntype{P}[1]{>{\raggedright\tabularxbackslash}p{#1}}
\newcommand{\specialcell}[2][c]{%
  \begin{tabular}[#1]{@{}c@{}}#2\end{tabular}}
\newcommand{\specialcelll}[2][l]{%
  \begin{tabular}[#1]{@{}l@{}}#2\end{tabular}}

\captionsetup[table]{aboveskip=0pt}
\captionsetup[table]{belowskip=10pt}

\linespread{1.2}

\usepackage{framed, color}
\definecolor{shadecolor}{RGB}{248,248,248}

\renewcommand{\textfraction}{0.05}
\renewcommand{\topfraction}{0.8}
\renewcommand{\bottomfraction}{0.8}
\renewcommand{\floatpagefraction}{0.75}

\renewenvironment{quote}{\begin{VF}}{\end{VF}}
\let\oldhref\href
\renewcommand{\href}[2]{#2\footnote{\url{#1}}}

\ifxetex
  \usepackage{letltxmacro}
  \setlength{\XeTeXLinkMargin}{1pt}
  \LetLtxMacro\SavedIncludeGraphics\includegraphics
  \def\includegraphics#1#{% #1 catches optional stuff (star/opt. arg.)
    \IncludeGraphicsAux{#1}%
  }%
  \newcommand*{\IncludeGraphicsAux}[2]{%
    \XeTeXLinkBox{%
      \SavedIncludeGraphics#1{#2}%
    }%
  }%
\fi

\makeatletter
\newenvironment{kframe}{%
\medskip{}
\setlength{\fboxsep}{.8em}
 \def\at@end@of@kframe{}%
 \ifinner\ifhmode%
  \def\at@end@of@kframe{\end{minipage}}%
  \begin{minipage}{\columnwidth}%
 \fi\fi%
 \def\FrameCommand##1{\hskip\@totalleftmargin \hskip-\fboxsep
 \colorbox{shadecolor}{##1}\hskip-\fboxsep
     % There is no \\@totalrightmargin, so:
     \hskip-\linewidth \hskip-\@totalleftmargin \hskip\columnwidth}%
 \MakeFramed {\advance\hsize-\width
   \@totalleftmargin\z@ \linewidth\hsize
   \@setminipage}}%
 {\par\unskip\endMakeFramed%
 \at@end@of@kframe}
 \@ifundefined{Shaded}{
}{\renewenvironment{Shaded}{\begin{kframe}}{\end{kframe}}}
\makeatother

% \renewenvironment{Shaded}{\begin{kframe}}{\end{kframe}}


\usepackage{makeidx}
\makeindex

\urlstyle{tt}

\usepackage{amsthm}
\makeatletter
\def\thm@space@setup{%
  \thm@preskip=8pt plus 2pt minus 4pt
  \thm@postskip=\thm@preskip
}
\makeatother

\frontmatter
\usepackage{booktabs}
\usepackage{longtable}
\usepackage{array}
\usepackage{multirow}
\usepackage{wrapfig}
\usepackage{float}
\usepackage{colortbl}
\usepackage{pdflscape}
\usepackage{tabu}
\usepackage{threeparttable}
\usepackage{threeparttablex}
\usepackage[normalem]{ulem}
\usepackage{makecell}
\usepackage{xcolor}
\usepackage[]{natbib}
\bibliographystyle{apalike}

\title{통계 프로그래밍 언어}
\author{한국한의학연구원, 구본초}
\date{2020-03-12}

\begin{document}
\maketitle

{
\hypersetup{linkcolor=}
\setcounter{tocdepth}{2}
\tableofcontents
}
\listoftables
\listoffigures
\hypertarget{overview}{%
\chapter*{Course Overview}\label{overview}}


본 문서는 2020년도 1학기 정보통계학과에서 개설한 ``통계 프로그래밍 언어'' 강의를 위해 개발한 강의 노트이며, Yihui Xie가 개발한 \textbf{bookdown} 패키지 \citep{R-bookdown}를 활용하여 생성한 문서이고 Google Chrome 또는 Firefox 브라우저에 최적화 됨. 아울러 충남대학교 정보통계학과 이상인 교수님의 2019년도 2학기 ``통계패키지활용'' 강의 노트와 동국대학교 ICT빅데이터학부 김진석 교수님의 \href{http://datamining.dongguk.ac.kr/lectures/R/_book/index.html}{R 프로그래밍 및 실습} 강의 자료 내용을 참고함.

본 강의 노트는 주 단위로 업데이트될 예정임.

\hypertarget{intro-lec}{%
\subsection*{강의소개}\label{intro-lec}}


\begin{quote}
R은 뉴질랜드 오클랜드 대학의 Robert Gentleman 과 Ross Ihaka 가 AT\&T 벨 연구소에서 개발한 S 언어를 기반으로 개발한 GNU 환경의 통계 계산 및 프로그래밍 언어이다. 현재 R 소프트웨어는 통계학 뿐 아니라 데이터 과학을 포함한 의학, 생물학 등 다양한 분야에서 활용되고 있으며 특히 통계 소프트웨어 개발과 데이터 분석에 많이 활용되고 있다. 본 강의는 데이터 분석을 위한 R의 기초 문법과 통계학 입문에서 학습한 몇 가지 중요한 통계적 이론에 대한 시뮬레이션 방법을 다룬다. 아울러 R package를 활용한 데이터 헨들링 및 시각화 그리고 Rmarkdown을 활용한 재현가능(reproducible)한 문서 작성법에 대해 학습하고자 한다.
\end{quote}

\hypertarget{purpose-course}{%
\subsection*{교과 목표}\label{purpose-course}}


\begin{quote}
\begin{itemize}
\tightlist
\item
  \textbf{R 기초 문법 습득}
\item
  \textbf{R package를 활용한 데이터 핸들링 및 자료 시각화}
\item
  \textbf{R 시뮬레이션을 통한 통계학 기초 이론 확인}
\item
  \textbf{R을 이용한 데이터 분석 실습}
\item
  \textbf{R markdown을 이용한 재현가능(reproducible)한 보고서 작성 방법 습득}
\end{itemize}
\end{quote}

\hypertarget{pre-course}{%
\subsection*{선수과목}\label{pre-course}}


\begin{quote}
\textbf{통계학 개론}
\end{quote}

\hypertarget{course-method}
\item
  \textbf{실험/실습: 50\%}
\end{itemize}

\hypertarget{grade-method}
\item
  \textbf{기말고사: 40 \%}
\item
  \textbf{출석: 10 \%}
\item
  \textbf{과제: 10 \%}
\end{itemize}
\end{quote}

\hypertarget{policy-course}{%
\subsection*{수업 규정}\label{policy-course}}


\begin{quote}
\begin{itemize}
\tightlist
\item
  3번 지각은 1번 결석으로 처리
\item
  특별한 사유 없이 수업 중간에 이탈한 경우 결석으로 처리
\item
  특별한 사유로 인해 결석 또는 지각을 할 경우 사유를 증빙할 수 있는 서류 제출 시 출석으로 인정
\item
  출결 미달, 중간 또는 기말고사 미 응시인 경우 F 학점으로 처리
\item
  수업 중 휴대폰 및 각종 모바일 기기 사용 금지
\end{itemize}
\end{quote}

\hypertarget{material-course}{%
\subsection*{교재 및 참고문헌}\label{material-course}}


\begin{quote}
별도의 교재 없이 본 강의 노트로 수업을 진행할 예정이며, 수업의 이해도 향상을 위해 아래 소개할 도서 및 웹 문서 등을 참고할 것을 권장함.
\end{quote}

\textbf{참고 자료}

\begin{itemize}
\tightlist
\item
  실리콘밸리 데이터과학자가 알려주는 따라하며 배우는 데이터 과학 \citep{kwon-2017}
\item
  R을 이용한 데이터 처리\&분석 \citep{seo-2014}
\item
  R 그래픽스 \citep{ryu-2005}
\item
  \href{https://ggplot2-book.org/}{ggplot2: elegant graphics for data analysis} \citep{wickham-2016}
\item
  \href{https://r4ds.had.co.nz/}{R for data science} \citep{wickham-2016r}
\item
  Statistical Computing with R \citep{rizzo-2019}
\end{itemize}

\hypertarget{course-schedule}{%
\subsection*{강의 계획}\label{course-schedule}}


\begin{table}[H]

\caption{\label{tab:make-schedule-tab}강의 계획표}
\centering
\fontsize{10}{12}\selectfont
\begin{tabular}[t]{c>{\raggedright\arraybackslash}p{6cm}c}
\toprule
주차 & 강의 내용 & 과제\\
\midrule
\rowcolor{gray!6}  Week 1 & R 소개, R/R Studio 설치, R 패키지 설치, R 맛보기 및 markdown 문서 만들기 & 과제 1\\
Week 2 & R 자료형: 스칼라, 벡터, 리스트 & \\
\rowcolor{gray!6}  Week 3 & R 자료형: 행렬 및 배열 & 과제 2\\
Week 4 & R 자료형: 팩터, 테이블, 데이터 프레임 & \\
\rowcolor{gray!6}  Week 5 & R 자료형: 문자열과 정규 표현식 & 과제 3\\
\addlinespace
Week 6 & 데이터 프레임 가공 및 시각화 I & \\
\rowcolor{gray!6}  Week 7 & 데이터 프레임 가공 및 시각화 II & 과제 4\\
Week 8 & 중간고사 & \\
\rowcolor{gray!6}  Week 9 & 데이터 프레임 가공 및 시각화 III & \\
Week 10 & R 프로그래밍: 조건문, 반복문, 함수 & 과제 5\\
\addlinespace
\rowcolor{gray!6}  Week 11 & 통계시뮬레이션 I: 표본분포 및 중심극한정리 & \\
Week 12 & 통계시뮬레이션 2: 신뢰구간과 가설검정 & 과제 6\\
\rowcolor{gray!6}  Week 13 & R을 이용한 기초통계 분석 & \\
Week 14 & R markdown 활용 & 과제 7\\
\rowcolor{gray!6}  Week 15 & 기말고사 & \\
\bottomrule
\end{tabular}
\end{table}

\mainmatter

\hypertarget{part-get-started}{%
\part{Get Started}\label{part-get-started}}

\hypertarget{intro-chap}{%
\chapter{Introduction}\label{intro-chap}}

\textbf{1. R프로그램}

\begin{itemize}
\tightlist
\item
  데이터 분석을 위한 자료 전처리, 통계 및 시각화를 지원하는 컴퓨터 언어 및 환경
\item
  1980년 AT\&T 벨 연구소의 John Chambers가 개발한 S 언어를 기반으로 1995년 뉴질랜드 Auckland 대학의 통계학과 교수 Robert Gentleman과 Ross Ihaka 가 개발
\item
  \href{https://en.wikipedia.org/wiki/GNU_Project}{GNU} 기반의 오픈 소스
\item
  통계학, 전산학, 생물학, 의학 등 거의 모든 학문분야에서 분석도구로 활용되고 있고, 최근 data science 분야에서 널리 활용
\end{itemize}

\textbf{2. R 언어의 특징}

\begin{itemize}
\tightlist
\item
  \textbf{무료 소프트웨어}
\item
  \href{http://cran.r-project.org/web/view}{CRAN (Comprehensive R Archive Network)}에서 배포
\item
  특정 vendor가 아닌 전 세계 연구자들이 개발한 알고리즘 및 최신 함수 활용 가능(packaging system)
\item
  범용적으로 사용되는 거의 대부분의 운영체제(Windows, Mac, Linux)에서 작동 가능
\item
  방대한 개발 및 사용 생태계 형성
\item
  강력한 그래픽 기능
\end{itemize}

\begin{quote}
\textbf{Tips: 유용한 웹 사이트}

\begin{itemize}
\tightlist
\item
  R과 관련한 거의 모든 문제는 Googling (구글을 이용한 검색)을 통해 해결 가능
\item
  다음 사이트는 꼭 즐겨찾기 추가!!

  \begin{itemize}
  \tightlist
  \item
    \href{https://stackoverflow.com}{Stack Overflow}
  \item
    \href{https://rpubs.com/}{RPubs}
  \end{itemize}
\end{itemize}
\end{quote}

\hypertarget{installation}{%
\section{R 설치하기}\label{installation}}

R 다운로드 사이트: \url{https://www.r-project.org} 또는 \url{https://cran.r-project.org}

\begin{enumerate}
\def\labelenumi{\arabic{enumi}.}
\item
  웹 브라우저(i.e.~Explore, Chrome, Firefox 등)의 주소 입력창에 \url{https://www.r-project.org}
\item
  좌측 R Logo 하단 Download 아래 CRAN 클릭
\item
  클릭 후 연결한 페이지를 스크롤 후 Korea 아래 링크\footnote{해당 링크들은 접속 시점에 따라 변경될 수 있음} 클릭
\item
  클릭 후 세 가지 운영체제(Linux, Mac OS X, Windowns)에 따른 R 버전 선택 가능\footnote{본 노트는 Windows 버전 설치만 다룸}
\item
  \textbf{Downloads R for Windows} 링크 클릭하면 다음과 같은 화면으로 이동
\item
  위 화면에서 \textbf{base} 링크 클릭 후 아래 화면에서 \textbf{Downloads R 3.4.2 for Windows} 를 클릭 후 설치 파일을 임의의 디렉토리에 저장 및 실행
\end{enumerate}

\begin{quote}
\textbf{Tip}: 3개 subdirectories에 대한 간략 설멍

\begin{itemize}
\tightlist
\item
  \texttt{base}: R 실행 프로그램
\item
  \texttt{contrib}: R package의 바이너리 파일
\item
  \texttt{Rtools}: R package 개발 및 배포를 위한 프로그램
\end{itemize}
\end{quote}

  \bibliography{book.bib,packages.bib}

\end{document}
